% Contributions are much appreciated, in order to contribute to this project, head over to this repository:
% https://github.com/bshramin/uofa-eng-assignment

\documentclass[11pt,letterpaper]{article}
\textwidth 6.5in
\textheight 9.in
\oddsidemargin 0in
\headheight 0in
\usepackage{graphicx}
\usepackage{fancybox}
\usepackage[utf8]{inputenc}
\usepackage{epsfig,graphicx}
\usepackage{multicol,pst-plot}
\usepackage{pstricks}
\usepackage{amsmath}
\usepackage{amsfonts}
\usepackage{amssymb}
\usepackage{eucal}
\usepackage[left=2cm,right=2cm,top=2cm,bottom=2cm]{geometry}
\pagestyle{empty}
\DeclareMathOperator{\tr}{Tr}
\newcommand*{\op}[1]{\check{\mathbf#1}}
\newcommand{\bra}[1]{\langle #1 |}
\newcommand{\ket}[1]{| #1 \rangle}
\newcommand{\braket}[2]{\langle #1 | #2 \rangle}
\newcommand{\mean}[1]{\langle #1 \rangle}
\newcommand{\opvec}[1]{\check{\vec #1}}
\renewcommand{\sp}[1]{$${\begin{split}#1\end{split}}$$}

\usepackage{lipsum}

\usepackage{listings}
\usepackage{color}

\definecolor{codegreen}{rgb}{0,0.6,0}
\definecolor{codegray}{rgb}{0.5,0.5,0.5}
\definecolor{codepurple}{rgb}{0.58,0,0.82}
\definecolor{backcolour}{rgb}{0.95,0.95,0.92}

\lstdefinestyle{mystyle}{
	backgroundcolor=\color{backcolour},   
	commentstyle=\color{codegreen},
	keywordstyle=\color{magenta},
	numberstyle=\tiny\color{codegray},
	stringstyle=\color{codepurple},
	basicstyle=\footnotesize,
	breakatwhitespace=false,         
	breaklines=true,                 
	captionpos=b,                    
	keepspaces=true,                 
	numbers=left,                    
	numbersep=5pt,                  
	showspaces=false,                
	showstringspaces=false,
	showtabs=false,                  
	tabsize=2
}

\lstset{style=mystyle}

\begin{document}
\pagestyle{plain}

\begin{flushleft}
Estudiante: Fabio Quimbay\\
Email: fabio.quimbay883@comunidadunir.net\\
Profesor: Miguel Ángel Cabeza\\
\end{flushleft}

\begin{flushright}\vspace{-20mm}
\includegraphics[height=2cm]{logo.png}
\end{flushright}
 
\begin{center}\vspace{0cm}
\textbf{\large PER5786 2022-2023  Física 1 (GFI) - PER5786 2022-2023}\\
 Tema 1 - Magnitudes y unidades físicas
\end{center}

 
\rule{\linewidth}{0.1mm}
%%%%%%%%%%%%%%%%%%%%%%%%%%%%%%%%%%%%%%%%%%%%%%%%%%%%%%%%%%%%%%%%%%%%%%%%

\bigskip
\bigskip

%%%%%%%%%%%%%%%%%%%%
\textbf{Ejercicio 2 propuesto}\\

Existe un parámetro en astrodinámica que se denomina "Parámetro de Tisserand" en honor al astrofísico francés que lo propuso. Una explicación breve de cómo se calcula la podéis ver en los siguientes enlaces:

\begin{enumerate}

	\item YouTube: https://www.youtube.com/watch?v=aZQIyR0pTsI
	\item Wikipedia: https://es.wikipedia.org/wiki/Parámetro\_de\_Tisserand
	
\end{enumerate}	

De acuerdo con el análisis dimensional de la ecuación para calcular el parámetro de Tisserand, ¿sería correcto afirmar que dicho parámetro es una magnitud física?\\

\textbf{Solución:}\\

El parámetro de Tisserand es una constante en el marco del problema circular restringido de tres cuerpos dado por \cite{WF}:

\begin{equation}\label{eq:1}
T = \frac{a_{2}}{a_{3}}+2\sqrt{\frac{a_{3}}{a_{2}}(1-{e_{3}}^2)}\cos{i_{3}}
\end{equation}

donde $a_{2}$ es el semieje mayor de la órbita del segundo cuerpo y $a_{3}$, $e_{3}$, $i_{3}$, respectivamente, el semieje mayor, la excentricidad y la inclinación de la órbita del tercer cuerpo.

Al expresar la ecuación $\ref{eq:1}$ en términos de sus dimensiones, obtenemos:

\begin{equation}\label{eq:2}
T = \frac{[L]}{[L]}+2\sqrt{\frac{[L]}{[L]}(1-{\frac{[L]^2}{[L]^2}})}1
\end{equation}

Simplificando la ecuación $\ref{eq:2}$, se identifica que las longitudes se cancelan quedando como valor una constante, sujeta a los parámetros empleados, por lo que \textbf{el Parámetro de Tisserand es adimensional}.

%%%%%%%%%%%%%%%%%%%%

\begin{thebibliography}{1}
 \bibitem[Wolffram]{WF} GALLARDO, Tabaré., Surfaces of Constant Tisserand Parameter., https://demonstrations.wolfram.com/SurfacesOfConstantTisserandParameter/
\end{thebibliography}

\end{document}

