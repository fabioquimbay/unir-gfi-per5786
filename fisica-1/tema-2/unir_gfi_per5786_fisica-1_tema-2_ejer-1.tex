% Contributions are much appreciated, in order to contribute to this project, head over to this repository:
% https://github.com/bshramin/uofa-eng-assignment

\documentclass[11pt,letterpaper]{article}
\textwidth 6.5in
\textheight 9.in
\oddsidemargin 0in
\headheight 0in
\usepackage{graphicx}
\usepackage{fancybox}
\usepackage[utf8]{inputenc}
\usepackage{epsfig,graphicx}
\usepackage{multicol,pst-plot}
\usepackage{pstricks}
\usepackage{amsmath}
\usepackage{amsfonts}
\usepackage{amssymb}
\usepackage{eucal}
\usepackage[left=2cm,right=2cm,top=2cm,bottom=2cm]{geometry}
\usepackage{esvect}
\pagestyle{empty}
\DeclareMathOperator{\tr}{Tr}
\newcommand*{\op}[1]{\check{\mathbf#1}}
\newcommand{\bra}[1]{\langle #1 |}
\newcommand{\ket}[1]{| #1 \rangle}
\newcommand{\braket}[2]{\langle #1 | #2 \rangle}
\newcommand{\mean}[1]{\langle #1 \rangle}
\newcommand{\opvec}[1]{\check{\vec #1}}
\renewcommand{\sp}[1]{$${\begin{split}#1\end{split}}$$}

\usepackage{lipsum}

\usepackage{listings}
\usepackage{color}

\definecolor{codegreen}{rgb}{0,0.6,0}
\definecolor{codegray}{rgb}{0.5,0.5,0.5}
\definecolor{codepurple}{rgb}{0.58,0,0.82}
\definecolor{backcolour}{rgb}{0.95,0.95,0.92}

\lstdefinestyle{mystyle}{
	backgroundcolor=\color{backcolour},   
	commentstyle=\color{codegreen},
	keywordstyle=\color{magenta},
	numberstyle=\tiny\color{codegray},
	stringstyle=\color{codepurple},
	basicstyle=\footnotesize,
	breakatwhitespace=false,         
	breaklines=true,                 
	captionpos=b,                    
	keepspaces=true,                 
	numbers=left,                    
	numbersep=5pt,                  
	showspaces=false,                
	showstringspaces=false,
	showtabs=false,                  
	tabsize=2
}

\lstset{style=mystyle}

\begin{document}
\pagestyle{plain}

\begin{flushleft}
Estudiante: Fabio Quimbay\\
Email: fabio.quimbay883@comunidadunir.net\\
Profesor: Miguel Ángel Cabeza\\
Fecha: Noviembre 3 de 2022\\
\end{flushleft}

\begin{flushright}\vspace{-20mm}
\includegraphics[height=2cm]{logo.png}
\end{flushright}
 
\begin{center}\vspace{0cm}
\textbf{\large PER5786 2022-2023  Física 1 (GFI) - PER5786 2022-2023}\\
 Tema 2 - Cinemática
\end{center}

 
\rule{\linewidth}{0.1mm}
%%%%%%%%%%%%%%%%%%%%%%%%%%%%%%%%%%%%%%%%%%%%%%%%%%%%%%%%%%%%%%%%%%%%%%%%

\bigskip
\bigskip

%%%%%%%%%%%%%%%%%%%%
\textbf{Ejercicio 1 propuesto}\\

El automóvil Thrust SSC en 1997 fue el primero en superar oficialmente la velocidad del sonido; estaba propulsado por dos motores a reacción.\\
El conductor realizó dos trayectos a lo largo del recorrido, uno en cada dirección, para anular los efectos del viento. El automóvil viajó primero de izquierda a derecha, y recorrió una distancia de 1609 m en un tiempo de 4.740 s. De derecha a izquierda, el automóvil recorrió la misma distancia en 4.695 s. A partir de estos datos, determina la velocidad media para cada trayecto.\\

\textbf{Solución Escenarios:}\\

Para el cálculo de la velocidad media se tomará la siguiente formula:

\begin{align}
\vec{V}_{m} = \frac{\Delta\vec{r}}{\Delta{t}}=\frac{\vec{r}(t_{2})-\vec{r}(t_{1})}{t_{2}-t_{1}}
\end{align}


\textbf{Trayecto No. 1}: $\qquad d=1,609\,m \qquad t_{1} = 0\,s \qquad t_{2} = 4.740\,s$

\begin{align}
\vec{V}_{m} = \frac{1,609\,m - 0\,m}{4.740\,s - 0 s} \Rightarrow 339.451\,m/s
\end{align}

La velocidad media $\vec{V}_{m}$ para el trayecto 1 es $\textbf{339.451\,m/s}$\\

\textbf{Trayecto No. 2}: $\qquad d=1,609\,m \qquad t_{1} = 0\,s \qquad t_{2} = 4.695\,s$

\begin{align}
\vec{V}_{m} = \frac{1,609\,m - 0\,m}{4.695\,s - 0 s} \Rightarrow 342.705\,m/s
\end{align}

La velocidad media $\vec{V}_{m}$ para el trayecto 2 es $\textbf{-342.705\,m/s}$.\footnote{El signo negativo se da debido el sentido de la trayectoria, de derecha a izquierda.}

%%%%%%%%%%%%%%%%%%%%

\end{document}

