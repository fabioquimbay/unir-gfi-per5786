% Contributions are much appreciated, in order to contribute to this project, head over to this repository:
% https://github.com/bshramin/uofa-eng-assignment

\documentclass[11pt,letterpaper]{article}
\textwidth 6.5in
\textheight 9.in
\oddsidemargin 0in
\headheight 0in
\usepackage{graphicx}
\usepackage{fancybox}
\usepackage[utf8]{inputenc}
\usepackage{epsfig,graphicx}
\usepackage{multicol,pst-plot}
\usepackage{pstricks}
\usepackage{amsmath}
\usepackage{amsfonts}
\usepackage{amssymb}
\usepackage{eucal}
\usepackage[left=2cm,right=2cm,top=2cm,bottom=2cm]{geometry}
\pagestyle{empty}
\DeclareMathOperator{\tr}{Tr}
\newcommand*{\op}[1]{\check{\mathbf#1}}
\newcommand{\bra}[1]{\langle #1 |}
\newcommand{\ket}[1]{| #1 \rangle}
\newcommand{\braket}[2]{\langle #1 | #2 \rangle}
\newcommand{\mean}[1]{\langle #1 \rangle}
\newcommand{\opvec}[1]{\check{\vec #1}}
\renewcommand{\sp}[1]{$${\begin{split}#1\end{split}}$$}

\usepackage{lipsum}

\usepackage{listings}
\usepackage{color}

\definecolor{codegreen}{rgb}{0,0.6,0}
\definecolor{codegray}{rgb}{0.5,0.5,0.5}
\definecolor{codepurple}{rgb}{0.58,0,0.82}
\definecolor{backcolour}{rgb}{0.95,0.95,0.92}

\lstdefinestyle{mystyle}{
	backgroundcolor=\color{backcolour},   
	commentstyle=\color{codegreen},
	keywordstyle=\color{magenta},
	numberstyle=\tiny\color{codegray},
	stringstyle=\color{codepurple},
	basicstyle=\footnotesize,
	breakatwhitespace=false,         
	breaklines=true,                 
	captionpos=b,                    
	keepspaces=true,                 
	numbers=left,                    
	numbersep=5pt,                  
	showspaces=false,                
	showstringspaces=false,
	showtabs=false,                  
	tabsize=2
}

\lstset{style=mystyle}

\begin{document}
\pagestyle{plain}

\begin{flushleft}
Estudiante: Fabio Quimbay\\
ID: 79883743
\end{flushleft}

\begin{flushright}\vspace{-15mm}
\includegraphics[height=2cm]{logo.png}
\end{flushright}
 
\begin{center}\vspace{0cm}
\textbf{\large PER5786 2022-2023  Cálculo (GFI) - PER5786 2022-2023}\\
Actividad 1. Boletín de problemas
\end{center}

 
\rule{\linewidth}{0.1mm}
%%%%%%%%%%%%%%%%%%%%%%%%%%%%%%%%%%%%%%%%%%%%%%%%%%%%%%%%%%%%%%%%%%%%%%%%

\bigskip
\bigskip

%%%%%%%%%%%%%%%%%%%%
\textbf{Objetivos de la actividad}\\

A través de este boletín de ejercicios sencillos podrás practicar los conocimientos adquiridos en los temas que van del 1 al 3.\\

\textbf{Descripcipción de la actividad}\\

Resuelve los siguientes problemas rápidos de manera individual. Para ello, puedes usar cualquier herramienta informática que permita la redacción de expresiones matemáticas, pero te recomendamos que uses estándares electrónicos tales como HTML, Markdown, LaTeX o LyX. Además de subir el documento al campus, también puedes aportar un enlace público a un repositorio. 

%%%%%%%%%%%%%%%%%%%%

\begin{enumerate}

	%%%%%%%%%%%%%%%%%%%%
	\item Simplifica la siguiente expresión con potencias:
	%%%%%%%%%%%%%%%%%%%%

	\begin{equation}
		\frac{a^3b^4c^7}{a^{-2}b^5\sqrt{(c)}}.
	\end{equation}

	Solución:

	%%%%%%%%%%%%%%%%%%%%
	\item Calcular el cociente de potencias:
	%%%%%%%%%%%%%%%%%%%%

	\begin{equation}
		\frac{2^33^2}{3^32}.
	\end{equation}

	Solución:
	
	%%%%%%%%%%%%%%%%%%%%
	\item Calcular:
	%%%%%%%%%%%%%%%%%%%%

	\begin{equation}
		\left( \frac{\left( 2 \frac{3}{9} : 3\right)^{-1}}{\left(\frac{9}{4}\right)^2\left(\frac{2}{5}\right)^{-1}} \right)
	\end{equation}

	Solución:

	%%%%%%%%%%%%%%%%%%%%
	\item Calcular y:
	%%%%%%%%%%%%%%%%%%%%

	\begin{equation}
		\left( \frac{\left( 2 \frac{3}{9} : 3\right)^{-1}}{\left(\frac{9}{4}\right)^2\left(\frac{2}{5}\right)^{-1}} \right)
	\end{equation}

	Solución:

\end{enumerate}


\end{document}

