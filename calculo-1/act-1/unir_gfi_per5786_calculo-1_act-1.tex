% Contributions are much appreciated, in order to contribute to this project, head over to this repository:
% https://github.com/bshramin/uofa-eng-assignment

\documentclass[11pt,letterpaper]{article}
\textwidth 6.5in
\textheight 9.in
\oddsidemargin 0in
\headheight 0in
\usepackage{graphicx}
\usepackage{fancybox}
\usepackage[utf8]{inputenc}
\usepackage{epsfig,graphicx}
\usepackage{multicol,pst-plot}
\usepackage{pstricks}
\usepackage{amsmath}
\usepackage{amsfonts}
\usepackage{amssymb}
\usepackage{eucal}
\usepackage[left=2cm,right=2cm,top=2cm,bottom=2cm]{geometry}
\pagestyle{empty}
\DeclareMathOperator{\tr}{Tr}
\newcommand*{\op}[1]{\check{\mathbf#1}}
\newcommand{\bra}[1]{\langle #1 |}
\newcommand{\ket}[1]{| #1 \rangle}
\newcommand{\braket}[2]{\langle #1 | #2 \rangle}
\newcommand{\mean}[1]{\langle #1 \rangle}
\newcommand{\opvec}[1]{\check{\vec #1}}
\renewcommand{\sp}[1]{$${\begin{split}#1\end{split}}$$}

\usepackage{lipsum}

\usepackage{listings}
\usepackage{color}

\definecolor{codegreen}{rgb}{0,0.6,0}
\definecolor{codegray}{rgb}{0.5,0.5,0.5}
\definecolor{codepurple}{rgb}{0.58,0,0.82}
\definecolor{backcolour}{rgb}{0.95,0.95,0.92}

\lstdefinestyle{mystyle}{
	backgroundcolor=\color{backcolour},   
	commentstyle=\color{codegreen},
	keywordstyle=\color{magenta},
	numberstyle=\tiny\color{codegray},
	stringstyle=\color{codepurple},
	basicstyle=\footnotesize,
	breakatwhitespace=false,         
	breaklines=true,                 
	captionpos=b,                    
	keepspaces=true,                 
	numbers=left,                    
	numbersep=5pt,                  
	showspaces=false,                
	showstringspaces=false,
	showtabs=false,                  
	tabsize=2
}

\lstset{style=mystyle}

\begin{document}
\pagestyle{plain}

\begin{flushleft}
	Estudiante: Fabio Quimbay\\
	Email: fabio.quimbay883@comunidadunir.net\\
	Profesor: José Luis Martínez Díaz\\
\end{flushleft}

\begin{flushright}\vspace{-15mm}
\includegraphics[height=2cm]{logo.png}
\end{flushright}
 
\begin{center}\vspace{0cm}
\textbf{\large PER5786 2022-2023  Cálculo (GFI) - PER5786 2022-2023}\\
Actividad 1. Boletín de problemas
\end{center}

 
\rule{\linewidth}{0.1mm}
%%%%%%%%%%%%%%%%%%%%%%%%%%%%%%%%%%%%%%%%%%%%%%%%%%%%%%%%%%%%%%%%%%%%%%%%

\bigskip
\bigskip

%%%%%%%%%%%%%%%%%%%%
\textbf{Objetivos de la actividad}\\

A través de este boletín de ejercicios sencillos podrás practicar los conocimientos adquiridos en los temas que van del 1 al 3.\\

\textbf{Descripcipción de la actividad}\\

Resuelve los siguientes problemas rápidos de manera individual. Para ello, puedes usar cualquier herramienta informática que permita la redacción de expresiones matemáticas, pero te recomendamos que uses estándares electrónicos tales como HTML, Markdown, LaTeX o LyX. Además de subir el documento al campus, también puedes aportar un enlace público a un repositorio. 

%%%%%%%%%%%%%%%%%%%%

\begin{enumerate}

	%%%%%%%%%%%%%%%%%%%%
	\item Simplifica la siguiente expresión con potencias: $\frac{a^3b^4c^7}{a^{-2}b^5\sqrt{(c)}}$
	%%%%%%%%%%%%%%%%%%%%

	Solución: $\frac{a^5 \cdot c^{\frac{13}{2}}}{b}$

	%%%%%%%%%%%%%%%%%%%%
	\item Calcular el cociente de potencias: $\frac{2^33^2}{3^32}$
	%%%%%%%%%%%%%%%%%%%%

	Solución: $\frac{2^2}{3} = \frac{4}{3}$
	
	%%%%%%%%%%%%%%%%%%%%
	\item Calcular: $\left( \frac{\left( 2 \frac{3}{9} : 3\right)^{-1}}{\left(\frac{9}{4}\right)^2\left(\frac{2}{5}\right)^{-1}} \right)$
	%%%%%%%%%%%%%%%%%%%%

	Solución:
	
	\begin{align}
		\frac{\left( 2 \frac{3}{9} : 3\right)^{-1}}{\left(\frac{9}{4}\right)^2\left(\frac{2}{5}\right)^{-1}}
		= \frac{\left(  \frac{2}{9} \right)^{-1}}{\left(\frac{9}{4}\right)^2\left(\frac{2}{5}\right)^{-1}}
		= \frac{\left(  \frac{2}{5} \right)}{\left(\frac{9}{4}\right)^2\left(\frac{2}{9}\right)}
		= \frac{\left(  \frac{2}{5} \right)}{\left(\frac{9}{4}\right)^2\left(\frac{1}{2}\right)}
		= \frac{16}{45}
		= \frac{2^4}{3^25}
	\end{align}

	%%%%%%%%%%%%%%%%%%%%
	\item Calcular y: $\log_{2}y^3 = 6$
	%%%%%%%%%%%%%%%%%%%%

	Solución:
	
	\begin{align*}
		\log_{2}y^3 &= 6 \Rightarrow \\  
		2^6 &= y^3 \\
		\sqrt[3]{2^6} &= \sqrt[3]{y^3} \\
		y &= 2^{\frac{6}{3}} = 2^2 = 4
	\end{align*}
	
	%%%%%%%%%%%%%%%%%%%%
	\item Sea $\log_{10}2 = 0,3010$, calcula el siguiente logaritmo: $\log_{10}\sqrt[4]{8}$
	%%%%%%%%%%%%%%%%%%%%

	Solución:

	\begin{align*}
		&= \log_{10}\sqrt[4]{8} \Rightarrow \log_{10}\sqrt[4]{2^3} = \log_{10}2^{\frac{3}{4}} = \frac{3}{4}\log_{10}2 = \frac{3}{4} \cdot (0.3010) = 0.22575
	\end{align*}

	%%%%%%%%%%%%%%%%%%%% 	
	\item Convierte los siguientes ángulos de radiantes a grados sexagesimales: $3\;rad, 2\pi/5\;rad, 3\pi/20$. 
	%%%%%%%%%%%%%%%%%%%%

	Solución:
	
	\begin{align*}
		\boxed{ r = g \cdot \frac{\pi}{180}\,radianes} \quad \boxed{ g = r \cdot \frac{180}{\pi}\,grados}
	\end{align*}
	
	De tal forma que:
	
	\begin{align}
		a)\,3\,rad \Rightarrow g = 3 \cdot \frac{180}{\pi} grados = \frac{580}{\pi} = 171.887^{\circ}\\
		b)\,\frac{2 \cdot \pi}{5}\,rad \Rightarrow g = \frac{2 \cdot \pi}{5} \cdot \frac{180}{\pi} grados = 2 \cdot 36 = 72^{\circ}\\
		c)\,\frac{3 \cdot \pi}{20}\,rad \Rightarrow g = \frac{3 \cdot \pi}{20} \cdot \frac{180}{\pi} grados = 3 \cdot 9 = 27^{\circ}
	\end{align}
	
	%%%%%%%%%%%%%%%%%%%%
	\item Sabiendo que $cos\,\alpha=\frac{1}{4}$ y que el ángulo está en el primer cuadrante, calcular las restantes razones trigonométricas para dicho ángulo.
	%%%%%%%%%%%%%%%%%%%%

	Solución:
	
	Dada la ecuación $cos\,\alpha = \frac{1}{4}$ y de acuerdo al teorema de pitágoras para triángulos rectángulos $a^2 = b^2 + c^2$ en donde $\textbf{b}$ y $\textbf{c}$ 
	corresponden a los catetos y $\textbf{a}$ a la hipotenusa, podemos conocer que en la ecuación original el 1 corresponde al cateto $\textbf{a} = 3$ y la hipotenusa $\textbf{c} = 4$; 
	por lo que es fácilmente poder determinar el valor del otro cateto, a saber $\textbf{b} = 3$. \\
	
	De tal forma podemos expresar las razones trigonométricas como siguen:
	
	\begin{itemize}
		\item $cos\,\alpha = \frac{1}{4}$
		\item $sec\,\alpha = \frac{1}{cos\,\alpha} = \frac{1}{\frac{1}{4}} = 4$
		\item $sen\,\alpha = \frac{b}{a} = \frac{3}{4}$
		\item $tan\,\alpha = \frac{sen\,\alpha}{cos\,\alpha} = \frac{\frac{3}{4}}{\frac{1}{4}} = \frac{12}{4} = 3$
		\item $csc\,\alpha = \frac{1}{sen\,\alpha} = \frac{a}{b} = \frac{3}{4}$
		\item $cot\,\alpha = \frac{1}{tan\,\alpha} = \frac{cos\,\alpha}{sen\,\alpha} = \frac{c}{b} = \frac{1}{3}$
	\end{itemize}

	%%%%%%%%%%%%%%%%%%%%
	\item Calcula la altura de una torre de refrigeración de una central nuclear sabiendo que la sombra mide 271 metros cuando los rayos solaren forman un ángulo de $30^\circ$.
	%%%%%%%%%%%%%%%%%%%%

	Solución:

	%%%%%%%%%%%%%%%%%%%%
	\item Calcula sin usar la calculadora ni tablas trigonométricas: $cos\;5\pi/12$, $cos\;7\pi/6$.
	%%%%%%%%%%%%%%%%%%%%

	Solución:

	%%%%%%%%%%%%%%%%%%%%
	\item Calcular el seno, el coseno y la tangente de $105^\circ$ en función del ángulo $210^\circ$. 
	%%%%%%%%%%%%%%%%%%%%

	Solución:

\end{enumerate}


\end{document}

