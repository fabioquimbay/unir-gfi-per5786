% Contributions are much appreciated, in order to contribute to this project, head over to this repository:
% https://github.com/bshramin/uofa-eng-assignment

\documentclass[11pt,letterpaper]{article}
\textwidth 6.5in
\textheight 9.in
\oddsidemargin 0in
\headheight 0in
\usepackage{graphicx}
\usepackage{fancybox}
\usepackage[utf8]{inputenc}
\usepackage{epsfig,graphicx}
\usepackage{multicol,pst-plot}
\usepackage{pstricks}
\usepackage{amsmath}
\usepackage{amsfonts}
\usepackage{amssymb}
\usepackage{eucal}
\usepackage[left=2cm,right=2cm,top=2cm,bottom=2cm]{geometry}
\pagestyle{empty}
\DeclareMathOperator{\tr}{Tr}
\newcommand*{\op}[1]{\check{\mathbf#1}}
\newcommand{\bra}[1]{\langle #1 |}
\newcommand{\ket}[1]{| #1 \rangle}
\newcommand{\braket}[2]{\langle #1 | #2 \rangle}
\newcommand{\mean}[1]{\langle #1 \rangle}
\newcommand{\opvec}[1]{\check{\vec #1}}
\renewcommand{\sp}[1]{$${\begin{split}#1\end{split}}$$}

\usepackage{lipsum}

\usepackage{listings}
\usepackage{color}
\usepackage{wrapfig}

\definecolor{codegreen}{rgb}{0,0.6,0}
\definecolor{codegray}{rgb}{0.5,0.5,0.5}
\definecolor{codepurple}{rgb}{0.58,0,0.82}
\definecolor{backcolour}{rgb}{0.95,0.95,0.92}

\lstdefinestyle{mystyle}{
	backgroundcolor=\color{backcolour},   
	commentstyle=\color{codegreen},
	keywordstyle=\color{magenta},
	numberstyle=\tiny\color{codegray},
	stringstyle=\color{codepurple},
	basicstyle=\footnotesize,
	breakatwhitespace=false,         
	breaklines=true,                 
	captionpos=b,                    
	keepspaces=true,                 
	numbers=left,                    
	numbersep=5pt,                  
	showspaces=false,                
	showstringspaces=false,
	showtabs=false,                  
	tabsize=2
}

\lstset{style=mystyle}

\begin{document}
\pagestyle{plain}

\begin{center}\vspace{0cm}
\textbf{\large PER5786 2022-2023  Física (GFI) - PER5786 2022-2023}\\
Actividad 1. Camino óptimo para búsqueda y rescate de un objetivo "invisible"
\end{center}

\begin{flushleft}
	\textbf{Profesor}: Miguel Ángel Cabeza\\
	\textbf{Fecha}: Diciembre XX de 2022\\
	\textbf{Estudiantes (Email)}: \\
	\quad Ferran Arnau Sánchez ()\\
	\quad Ivanna Delgado Irigoyen ()\\
	\quad Jessika Andrea Aceituno Dardon ()\\
	\quad Miguel Julián Lanuza ()\\
	\quad Fabio Quimbay (fabio.quimbay883@comunidadunir.net)
\end{flushleft}

\begin{flushright}\vspace{-35mm}
\includegraphics[height=3cm]{logo.png}
\end{flushright}

\rule{\linewidth}{0.1mm}
%%%%%%%%%%%%%%%%%%%%%%%%%%%%%%%%%%%%%%%%%%%%%%%%%%%%%%%%%%%%%%%%%%%%%%%%

\bigskip
\bigskip

%%%%%%%%%%%%%%%%%%%%
\textbf{Objetivos de la actividad}\\

xxx\\

\textbf{Descripcipción de la actividad}\\

xxx

%%%%%%%%%%%%%%%%%%%%

\begin{enumerate}

	%%%%%%%%%%%%%%%%%%%%
	\item \textbf{Actividad 1}:
	
	Realiza la sustitución de la ecuación 2 en la ecuación 3 para llegar a obtener una expresión de $\theta(r)$. Una vez conseguida la ecuación de $\theta(r)$, 
	despeja r en función de $\theta$, y emplea la ecuación 1, para obtener una expresión de $\theta(r)$ dependiente de las variables iniciales del problema 
	$v_{1}, v_{2}, t_{c}$. Esta última ecuación (ecuación 4) será la que empleemos más adelante para dibujar gráficamente la parte no lineal del recorrido.

	\textbf{Solución}: 

	%%%%%%%%%%%%%%%%%%%%

	\item \textbf{Actividad 2}:
	
	Busca información para explicar qué funciones realizan las instrucciones iniciales clear y clf. Incluye cada explicación en cada línea de código.

	\textbf{Solución}: 
	
	%%%%%%%%%%%%%%%%%%%%

	\item \textbf{Actividad 3}:
	
	Tras definir las variables iniciales proporcionadas en el enunciado del problema, pasamos a definir aquellas variables intermedias necesarias para los cálculos; 
	para ello completa en el anterior código la definición de la variable D.

	\textbf{Solución}: 
	
	%%%%%%%%%%%%%%%%%%%%	
	
	\item \textbf{Actividad 4}:
	
	Busca información sobre cómo crear un vector en MatLab/Octave; deberá tener inicio en el valor 0, tener un paso entre valores de $delta_t$, y un valor final de $t_{i}$.

	\textbf{Solución}: 

	%%%%%%%%%%%%%%%%%%%%
	
	\item \textbf{Actividad 5}:
	
	¿Cuál es el valor máximo de theta en radianes? Completa el código de la imagen interior.

	\textbf{Solución}: 

	%%%%%%%%%%%%%%%%%%%%	
	
	\item \textbf{Actividad 6}:
	
	Con la ecuación obtenida en la Actividad 1, completa en el código anterior la expresión para calcular la distancia máxima desde el origen. Recuerda emplear el ángulo 
	máximo que antes hemos indicado.

	\textbf{Solución}: 

	%%%%%%%%%%%%%%%%%%%%		
	
	\item \textbf{Actividad 7}:
	
	Crea una variable vector que represente el tiempo en la maniobra angular, que comience en $t_{i}$, termine en $t_{max}$ y tenga por valor de incremento $delta_{t}$.
	Luego escribe la expresión de theta que obtuvimos en la Ecuación 3. Y finalmente, escribe la expresión de r que obtuviste como Ecuación 4 en la Actividad 1.

	\textbf{Solución}: 

	%%%%%%%%%%%%%%%%%%%%		
	
	\item \textbf{Actividad 8}:
	
	Crea una matriz numérica de dos columnas y las filas necesarias para representar la latitud y la longitud de las posiciones del recorrido completo.

	\textbf{Solución}: 

	%%%%%%%%%%%%%%%%%%%%	
	
	\item \textbf{Actividad 9}:
	
	Almacena la matriz anterior (mediante la función "writecsv") en un fichero llamado "coordenadas.csv" en el que se almacenarán las posiciones del barco de rescate
	en cada periodo temporal.

	\textbf{Solución}: 

	%%%%%%%%%%%%%%%%%%%%		
	
	\item \textbf{Actividad 10}:
	
	Convierte el fichero csv anteriormente generado en un formato KML compatible con Google Earth.

	\textbf{Solución}: 

	%%%%%%%%%%%%%%%%%%%%							

\end{enumerate}


\end{document}

