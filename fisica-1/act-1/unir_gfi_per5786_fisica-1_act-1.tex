% Contributions are much appreciated, in order to contribute to this project, head over to this repository:
% https://github.com/bshramin/uofa-eng-assignment

\documentclass[11pt,letterpaper]{article}
\textwidth 6.5in
\textheight 9.in
\oddsidemargin 0in
\headheight 0in
\usepackage{graphicx}
\usepackage{fancybox}
\usepackage[utf8]{inputenc}
\usepackage{epsfig,graphicx}
\usepackage{multicol,pst-plot}
\usepackage{pstricks}
\usepackage{amsmath}
\usepackage{amsfonts}
\usepackage{amssymb}
\usepackage{eucal}
\usepackage[left=2cm,right=2cm,top=2cm,bottom=2cm]{geometry}
\pagestyle{empty}
\DeclareMathOperator{\tr}{Tr}
\newcommand*{\op}[1]{\check{\mathbf#1}}
\newcommand{\bra}[1]{\langle #1 |}
\newcommand{\ket}[1]{| #1 \rangle}
\newcommand{\braket}[2]{\langle #1 | #2 \rangle}
\newcommand{\mean}[1]{\langle #1 \rangle}
\newcommand{\opvec}[1]{\check{\vec #1}}
\renewcommand{\sp}[1]{$${\begin{split}#1\end{split}}$$}

\usepackage{lipsum}

\usepackage{listings}
\usepackage{color}
\usepackage{wrapfig}

\definecolor{codegreen}{rgb}{0,0.6,0}
\definecolor{codegray}{rgb}{0.5,0.5,0.5}
\definecolor{codepurple}{rgb}{0.58,0,0.82}
\definecolor{backcolour}{rgb}{0.95,0.95,0.92}

\lstdefinestyle{mystyle}{
	backgroundcolor=\color{backcolour},   
	commentstyle=\color{codegreen},
	keywordstyle=\color{magenta},
	numberstyle=\tiny\color{codegray},
	stringstyle=\color{codepurple},
	basicstyle=\footnotesize,
	breakatwhitespace=false,         
	breaklines=true,                 
	captionpos=b,                    
	keepspaces=true,                 
	numbers=left,                    
	numbersep=5pt,                  
	showspaces=false,                
	showstringspaces=false,
	showtabs=false,                  
	tabsize=2
}

\lstset{style=mystyle}

\begin{document}
\pagestyle{plain}

\begin{center}\vspace{0cm}
\textbf{\large PER5786 2022-2023  Física (GFI) - PER5786 2022-2023}\\
Actividad 1. Camino óptimo para búsqueda y rescate de un objetivo "invisible"
\end{center}

\begin{flushleft}
	Profesor: Miguel Ángel Cabeza\\
	Fecha: Diciembre XX de 2022\\
	Estudiantes (Email): \\
	\begin{enumerate}
		\item Ferran Arnau Sánchez ()
		\item Ivanna Delgado Irigoyen ()
		\item Jessika Andrea Aceituno Dardon ()
		\item Miguel Julián Lanuza ()
		\item Fabio Quimbay (fabio.quimbay883@comunidadunir.net)
	\end{enumerate}
\end{flushleft}

\begin{wrapfigure}{l}{0.25\textwidth}
    \centering
    \includegraphics[width=0.25\textwidth]{logo.png}
\end{wrapfigure}

 
\rule{\linewidth}{0.1mm}
%%%%%%%%%%%%%%%%%%%%%%%%%%%%%%%%%%%%%%%%%%%%%%%%%%%%%%%%%%%%%%%%%%%%%%%%

\bigskip
\bigskip

%%%%%%%%%%%%%%%%%%%%
\textbf{Objetivos de la actividad}\\

xxx\\

\textbf{Descripcipción de la actividad}\\

xxx

%%%%%%%%%%%%%%%%%%%%

\begin{enumerate}

	%%%%%%%%%%%%%%%%%%%%
	\item \textbf{Actividad 1}:
	
	Realiza la sustitución de la ecuación 2 en la ecuación 3 para llegar a obtener una expresión de $\theta(r)$. Una vez conseguida la ecuación de $\theta(r)$, 
	despeja r en función de $\theta$, y emplea la ecuación 1, para obtener una expresión de $\theta(r)$ dependiente de las variables iniciales del problema 
	$v_{1}, v_{2}, t_{c}$. Esta última ecuación (ecuación 4) será la que empleemos más adelante para dibujar gráficamente la parte no lineal del recorrido.
	%%%%%%%%%%%%%%%%%%%%

	\textbf{Solución}: 


	\item \textbf{Actividad 2}:
	
	Busca información para explicar qué funciones realizan las instrucciones iniciales clear y clf. Incluye cada explicación en cada línea de código.
	%%%%%%%%%%%%%%%%%%%%

	\textbf{Solución}: 

	\item \textbf{Actividad 3}:
	
	Tras definir las variables iniciales proporcionadas en el enunciado del problema, pasamos a definir aquellas variables intermedias necesarias para los cálculos; 
	para ello completa en el anterior código la definición de la variable D.
	%%%%%%%%%%%%%%%%%%%%

	\textbf{Solución}: 
	
		\item \textbf{Actividad 4}:
	
	Busca información sobre cómo crear un vector en MatLab/Octave; deberá tener inicio en el valor 0, tener un paso entre valores de $delta_t$, y un valor final de $t_{i}$.
	%%%%%%%%%%%%%%%%%%%%

	\textbf{Solución}: 

\end{enumerate}


\end{document}

